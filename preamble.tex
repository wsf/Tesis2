%--------------------------------------------------------------------------------------------------------------------
%------------------------------------------------- Preamble ---------------------------------------------------------
\documentclass[a4paper,11pt,openright]{report}
\renewcommand{\familydefault}{\sfdefault}



\usepackage{xcolor}

%New colors defined below
\definecolor{codegreen}{rgb}{0,0.6,0}
\definecolor{codegray}{rgb}{0.5,0.5,0.5}
\definecolor{codepurple}{rgb}{0.58,0,0.82}
\definecolor{backcolour}{rgb}{0.95,0.95,0.92}
\usepackage{listings}

%Code listing style named "mystyle"
\lstdefinestyle{mystyle}{
  backgroundcolor=\color{backcolour}, commentstyle=\color{codegreen},
  keywordstyle=\color{magenta},
  numberstyle=\tiny\color{codegray},
  stringstyle=\color{codepurple},
  basicstyle=\ttfamily\footnotesize,
  breakatwhitespace=false,         
  breaklines=true,                 
  captionpos=b,                    
  keepspaces=true,                 
  numbers=left,                    
  numbersep=5pt,                  
  showspaces=false,                
  showstringspaces=false,
  showtabs=false,                  
  tabsize=2
}



\usepackage{framed}
\usepackage{color}
\definecolor{myblue}{rgb}{0,0,0.7}
\definecolor{mygrey}{rgb}{0.6,0.6,0.6}
\definecolor{myred}{rgb}{0.8,0,0}
\definecolor{myred2}{rgb}{0.7,0,0}
\usepackage[colorlinks=false,citecolor=myblue,linkcolor=myblue]{hyperref}
%\usepackage[center]{subfigure}
\usepackage{graphics,graphicx}
\usepackage[spanish]{babel}
\usepackage{newlfont}
\usepackage{amsthm}
\usepackage{amsfonts}
\usepackage{amssymb}
\usepackage{amsmath}
%\usepackage{mathdots}
\usepackage{booktabs}
\usepackage{longtable}
\usepackage{lscape}
\usepackage{multirow}
\usepackage{natbib}
\usepackage[utf8]{inputenc}
\usepackage{fancyhdr}
\usepackage[dvips]{graphicx}
%\usepackage{wrapfig} 
%\usepackage{fancyhdr,epsfig}
%\usepackage{colortbl}
%\usepackage{a4wide}
%\usepackage{amssymb}
%\usepackage[latin1]{inputenc}
%\usepackage[spanish]{babel}
\DeclareGraphicsExtensions{.bmp,.png,.pdf,.jpg,.eps}



\bibpunct{\textcolor{myblue}{(}}{\textcolor{myblue}{)}}{\textcolor{myblue}{;}}{a}{\textcolor{myblue}{,}}{\textcolor{myblue}{,}}


% ---- graphics file types -----

%\DeclareGraphicsExtensions{.png}

% ------- Fonts --------
%\usepackage[T1]{fontenc}
% Cambia el formato y posici{\'o}n de las divisiones del documento
%\usepackage{sectsty}
%\allsectionsfont{\sffamily\mdseries}

% ------------- Fancy Chapters ---------------
\usepackage[Sonny]{fncychap}
%Sonny, Lenny, Glenn, Conny, Rejne and Bjarne.
\ChNameVar{\fontencoding{T1}
  \fontfamily{cmss}
  \fontseries{m}
  \fontshape{n}
  \fontsize{30}{12}
  \selectfont}
\ChNumVar{\fontencoding{T1}
  \fontfamily{cmss}
  \fontseries{m}
  \fontshape{n}
  \fontsize{30}{12}
  \selectfont}
\ChTitleVar{\fontencoding{T1}
  \fontfamily{cmss}
  \fontseries{m}
  \fontshape{n}
  \fontsize{30}{12}
  \selectfont}

% ------------------------------ Page Layout ------------------------------------

\usepackage[a4paper,text={13cm,24cm},bindingoffset=2cm,top=3cm,twoside]{
geometry }


%\usepackage[a4paper,left=5cm,top=3cm,right=2cm,twoside]{geometry}

%---------------------- Fancy Header ------------------------
\usepackage{fancyhdr}
\pagestyle{fancy} \fancyhead{} \fancyfoot{}
\renewcommand{\sectionmark}[1]{\markright{\thesection.\ #1}}
%\fancyhead[R]{\textbf{\thepage}}
%\fancyhead[L]{\nouppercase{\textbf{\rightmark}}}
\fancyhead[LE,RO]{\nouppercase{\textbf{\rightmark}}}
\fancyhead[LO,RE]{}
\fancyfoot[LE,RO]{\textbf{\thepage}}
\fancyfoot[LO,RE]{}
\renewcommand{\headrulewidth}{0.3pt}
\renewcommand{\footrulewidth}{0.3pt}


% ----------------------------------- Draft Version Stamp --------------------------------------------
\usepackage[draft,time,scrtime]{prelim2e}
\renewcommand{\PrelimWords}{\normalsize\fontfamily{augie}\selectfont\color{myred}Versi{\'o}n Borrador: 1.0}
%
\renewcommand{\thefootnote}{\textrm{\alph{footnote}}}

%--- Tree Diagram ---
\usepackage{dirtree}
\DeclareGraphicsExtensions{.bmp,.png,.pdf,.jpg,.eps}
\usepackage{graphics,graphicx}
\usepackage{wrapfig} 

%\setglossgroup{A}{Abreviaciones}%
%\setglossgroup{S}{S{\'\i}mbolos}

%\newcommand{\mydrop}[1]{\setbox0\vbox{\noindent\fontsize{30}{30}\selectfont\noindent #1}
%\wd0=0pt\ht0=0pt\box0 \hangindent=6mm\hangafter =-4\vspace{-3mm} \normalsize\noindent}

%%%%%%%%%%%%%%%%%%%%%%%%%%%% Setting to control figure placement
% These determine the rules used to place floating objects like figures
% They are only guides, but read the manual to see the effect of each.
   \renewcommand{\topfraction}{.99}
   \renewcommand{\bottomfraction}{.99}
   \renewcommand{\textfraction}{.01}
%\parindent=1.3cm
%\parindent=2cm
%\parskip=5cm

\newtheorem{defi}{{\sc Definición}}
\newtheorem{ejemplo}{\rule{0.3in}{0.11in} {\rm Ejemplo }}
\textwidth=13cm

%-----------Formato Codigo----------
\usepackage{listings}
\lstset{basicstyle=\fontfamily{augie}}



