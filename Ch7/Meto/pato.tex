% This is LLNCSDE2.TEX, a variation of LLNCS.DEM
% (the demonstration file of

% the LaTeX macro package from Spring
%%Verlag
% for Lecture Notes in Computer Science,
% version 2.2 for LaTeX2e),

% which can be used by volume editors for the preparation
% of the front matter pages and the author index
%
% Last changes: 04.08.1999, Antje Endemann (endemann@springer.de)
%
%%%%%%%%%%%%%%%%%%%%%%%%%%%%%%%%%%%%%%%%%%%%%%%%%%%%%%%%%%%%%%%%%%%%%
% In order to generate an Author Index do the following:
% After TeXing this document start the program MakeIndex by typing
% MAKEINDX -S SPRMINDX.STY <filename>
% (generates an IND file for the Author Index)
% into the DOS command line.
% (At other systems you may have to use the command MAKEINDEX.)
% Now TeX this file once again, then you will get an Author Index.
% TeX this file once more, then the TOC will be complete.
%%%%%%%%%%%%%%%%%%%%%%%%%%%%%%%%%%%%%%%%%%%%%%%%%%%%%%%%%%%%%%%%%%%%%

\documentclass[11pt]{llncs}
%\documentclass[10pt]{article}
%\usepackage[latin1]{inputenc}
%\usepackage{ucs}
\usepackage[utf8]{inputenc}
\usepackage{makeidx}  % allows for indexgeneration
\usepackage[latin1]{inputenc}
%\usepackage{fancyhdr,epsfig}
\usepackage{graphicx}
\usepackage{a4wide}
\usepackage{listings}

\usepackage{amssymb}
\usepackage{enumerate}
\usepackage{ulem}
\usepackage{listings}
\usepackage{layout}

\usepackage{pslatex}
\usepackage[T1]{fontenc}
\usepackage[latin1]{inputenc}
%\usepackage{a4wide}

\renewcommand{\baselinestretch}{1}

\makeindex
%
\begin{document}
%
\frontmatter          % for the preliminaries
%
%%%%%%\setcounter{page}{5}
%
\pagestyle{headings}  % switches on printing of running heads
\addtocmark{Hamiltonian Mechanics} % additional mark in the TOC
%


\title{Contratos context-aware para e-learning}

\author{Alejandro Sartorio}

\index{Ekeland, Ivar} \index{Temam, Roger}

% use the command \index{<name>} for index entries

%\institute{Lifia. Facultad de Informatica, Universidad Nacional de La Plata \and Consejo Nacional de Investigaciones Cientficas
%y Tcnicas CONICET \and Facultad de Humanidades y Artes,Universidad Nacional de Rosario \and Gidis. Facultad de Ciencias Exactas
%Ingeniera y Agrimensura, Universidad Nacional de Rosario \\}


\institute{Facultad de Ciencias Exactas Ingeniera y Agrimensura, Universidad Nacional de Rosario \\}

\maketitle

\begin{abstract}
\end{abstract}

\begin{quote}

\small{\textbf{Palabras Claves:} Coordinacin de contratos, e-learning, "Context-Aware"}


\end{quote}



\section{Contratos}

Bajo que condicionamiento estar'ia dichos datos .....

habilitaci'on de servicios de los dispositivos hipermediales. 


Relaciones vol'atiles para poder poner el contrato.

Deficini'on de terminos para obra abierta.


Poderaci'on MCondicionales y DM condicionales

\begin{itemize}
\item contextos
\item reglas
\subitem condicionales
\subitem acciones
\end{itemize}








\setcounter{page}{13}

\addtocmark[2]{Author Index} % additional numbered TOC entry
\renewcommand{\indexname}{Author Index}
\printindex

\newpage
\begin{thebibliography}{1}
    \bibitem{ceri} Stefano Ceri, Florian Daniel, Maristella Matera, Federico M. Facca. 20YY. {Model-driven Development of 
\end{thebibliography}


\end{document}
