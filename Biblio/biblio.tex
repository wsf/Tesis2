%--------------------------------------------------------------------------------------------------------------------
%------------------------------------------------- Chapter 9 --------------------------------------------------------
%\chapter{Bibliografía}
%\pagenumbering{arabic}


\begin{thebibliography}{1}


\bibitem{cap1.1} 
Vicerrectoría de Investigaciones de la Universidad del Cauca. Informe Final del
Proyecto RedPacificoCyT. Popayán, Enero de 2002.


\bibitem{cap1.2}  
Jonathan Grudin. CSCW: History and Focus [en línea].
University of California. IEEE Computer, 27, 5, 19-26. 1994 [citado 2003-02-22].
Disponible en
la Web:\url{http://www.ics.uci.edu/~grudin/Papers/IEEE94/IEEEComplastsub.html}


\bibitem{cap1.3}
Jonathan Grudin. Groupware and social dynamics: eight challenges for
developers [en línea]. University of California. Communications of the ACM, 37,
1, 92-105. 1994 [citado 2003-02-22]. Disponible en la Web:
\url{http://www.ics.uci.edu/~grudin/Papers/CACM94/cacm94.html}

\bibitem{cap1.133}
HAYA, P. A., ALAMÁN, X., AND MONTORO, G. A comparative study of
communication infrastructures for the implementation of ubiquitous computing.
UPGRADE, The European Journal for the Informatics Professional
2, 5 (2001).

\bibitem{cap1.196}
NORMAN, D. The Invisible Computer. MIT Press, 1998.

\bibitem{cap1.4.groupware}
HSU, Jeffrey. Collaborative Computing: Byte Magazin. Dic. 1.988


\bibitem{cap1.3.groupware}
HERNANDEZ, Marcela y RODRIGUEZ Marcela. Trabajo en grupo y “groupware”.
Bogota CIFIUNIANDES, 1.996, p 6.

\bibitem{cap1.1.groupware} SCHNAIDT, Patricia. Workflow Applications. An
emerging Method for Sharing Work: LAN Magazine. Feb. 1992


\bibitem{cap1.5.groupware} HERNANDEZ, Marcela y RODRIGUEZ Marcela. Trabajo
en grupo y “groupware”. Bogota . CIFIUNIANDES, 1.996, p 6.



\bibitem {cacic2007.7} Distante  D., Tilley S. and Huang S. (2004b).
{Documenting
software systems with views IV:
documenting web transaction design with UWAT+.} P\textit{roceedings of the 22nd
International
Conference on Design of Communication (SIGDOC 2004), Memphis, TN, New York, NY:
ACM Press, 10–13 October.}

\bibitem {cacic2007.14}
Gelernter D. and Carriero N. {Coordination Languages and
their Significance}.\textit{ Communications ACM 35, 2, pp. 97-107, 1992.}


\bibitem {cacic2007.9}

Andrade L. and Fiadeiro J.L. {Interconnecting Objects via Contracts.} 
I\textit{n UML'99 – Beyond the Standard,
R.France and B.Rumpe (eds), LNCS 1723, Springer Verlag 1999, 566-583.}

\bibitem{} 
Anderson M., Ball M., Boley H., Greene S., Howse N., Lemire D., McGrath S.
(2003), RACOFI: A Rule-Applying Collaborative Filtering System. Pages 53–72
of Proc. COLA’03. IEEE/WIC. Halifax, Canada.

\bibitem{cap1.113}
GONZÁLEZ, J. L., RUBIO, A., AND MOLL, F. Human powered piezoelectric batteries
to supply power to wearable electronic devices. International Journal of
the Society of Materials Engineering for Resources 10,1 (2002), 33-40.


\bibitem{cap1.6}
ABOWD, G. D., ATKESON, C. G., HONG, J. L, LONG, S., KOOPER,
R., AND PlNKERTON, M. Cyberguide: A mobile context-aware tour guide.
Wireless Networks 3, 5 (October 1997), 421-433.

\bibitem{cap1.11}
AIRE, http://aire.csail.mit.edu/, 2003.

\bibitem{cap1.5}
ABOWD, G. D. Classroom 2000: an experiment with the instrumentation
of a living educational environment. IBM System Journal 38, 4 (1999),
508-530.

\bibitem{cap1.2}
ABASCAL, J., CAGIGAS, D., GARAY, N., AND GARDEAZABAL, L. MObile
interface for a smart wheelchair. In Fourth International Symposium
on Human Computer Interaction with Mobile Devices (Pisa (Italy), 2002),
F. Paterno, Ed., vol. 411 of LNCS, Springer-Verlag, pp. 373-377.

\bibitem{cap1.69}
CLARK, H. H. Using language. Cambridge University Press, 1996.


\bibitem{cap1.198}
OLIVER, N. Towards Perceptual Intelligence: Statistical Modeling of Human
Individual and Interactive Behaviors. PhD thesis, MIT Media Lab, 2000.

\bibitem{cap1.205}
PATERNO, F., AND SANTORO, C. One model, many interfaces. In
Computer-Aided Design of User Interfaces III. (Dordrecht, Hardbound, May
2002), C. Kolski and J. Vanderdonckt, Eds., GADUI, Kluwer Academic Publishers,
pp. 143-154.


\bibitem{cap1.16}
ALI, M. F., PÉREZ-QUIÑONES, M. A., ABRAMS, M., AND SHELL, E.
Building Multi-Platform User Interfaces with UIML, computer-aided design
of user interfaces iii (cadui) ed. Kluwer Academic Publishers, Dordrecht,
Hardbound, 2002, ch. 22, pp. 225-236.

\bibitem{cap1.214}
PUERTA, A., AND EISENSTEIN, J. XIML: A universal language for user
interfaces. White Paper, 20Ü1. http://www.ximl.org/Docs.asp.

\bibitem{cap1.20}
ALONSO, N., BALAGUERA, A., JUNYENT, E., LAFUENTE, A., LÓPEZ,
J. B., LORES, J., MUÑOZ, D., PÉREZ, M., AND TARTERA, E. Virtual
reality as an extensión ofthe archaeological record: The reconstruction ofthe
iron age fortress el vilars. In CAÁ, Computer Applications,and quantitative
methods in Archaeology (Ljubliana, Eslovenia, 2000).


\bibitem{UWA32}
UWA (2001), UWA Requirements Elicitation: Model, Notation, and Tool Architecture


\bibitem{UWA5} A. Dardenne, A. van Lamsweerde, and S. Fickas,
"Goal-directed Requirements Acquisition", Science of Computer Programming;
Vol. 20, 1993


\bibitem{HDM11}
Garzoto F., Schwabe D. and Paolini P. (1993) HDM-A Model Based Approach
to Hypermedia Aplication Design. ACM Trnasactions on Information System, 11 (1),
pp 1-26.

\bibitem{}
Baeza, Y. R. y Ribeiro, N. B. (1999) Modern Information Retrieval, Nueva Yotk,
Addison Wesley, ACM, disponible
en: http://www.dcc.ufmg.br/irbook/print/chap10.ps.gz


\bibitem{}
Balabanovic, M. y Shoham, Y. (1997) Fab: Content-based, collaborative
recommendation, Nueva York, Communications of the ACM, disponible en:
http://citeseer.ist.psu.edu/balabanovic97combining.html

\bibitem{SG96}
Mary Shaw, David Garlan. Software Architectures – Perspectives on an Emerging
Discipline, Prentice Hall, 1996.


\bibitem{Sha94}
Mary Shaw. Procedure Calls Are the Assembly Language of Software Intercon-
nection: Connectors Deserve First-Class Status. En Proceedings of Workshop on
Studies of Software Design, Enero 1994.



\bibitem{SDK95}
Mary Shaw, Robert DeLine, Daniel V. Klein, Theodore L. Ross, David M.
Young, Gregory Zelesnik. Abstractions for Software Architecture and Tools to Support
Them. IEEE Transactions on Software Engineering, vol. 21, no 4, págs. 314-
335, abril de 1995.


\bibitem{}
Bradshaw, S., and Hammond, K. J. (1999) Mining Citation Text as Indices
for Document, Medford Nueva York, en Proceedings of the ASIS 1999 Annual
Conference.

\bibitem{}
Brown, P. J. y Jones, G. J. F. (2002) Exploiting contextual change in
context-aware retrieval, Madrid, ACM Press, New York, Proceedings of the 17th
ACM Symposium on Applied Computing (SAC 2002), disponible
enhttp://portal.acm.org/citation.cfm?id=508917&coll=GUIDE&dl=GUIDE&CFID=16172620
&C FTOKEN=19042730


\bibitem{} 

Berners-Lee, Tim. (1998), A roadmap to the Semantic Web. disponible en:
http://www.w3.org/DesignIssues/Semantic.html.

\bibitem{} 
Brown, P. J., Jones, G.J.F., (2000) Context-aware retrieval: exploring a new
environment for information retrieval and information filtering. Personal and
Ubiquitous Computing, 5(4):253–263, 


\bibitem{} 
Brown, P., Burleston, W., Lamming, M., Rahlff, O., Romano, G., Scholtz, J., and
Budzik, J. y Hammond, K. (2000) User Interactions with Everyday Applications as
Context for Just-in-time Information Access. Proceedings of Intelligent User
Interfaces 2000. ACM, disponible en:
http://portal.acm.org/citation.cfm?id=325776&coll=GUIDE&dl=GUIDE&CFID=16172
620&CFTOKEN=19042730


\bibitem{} 
Cabero J. (2006) Bases pedagógicas del e-learning, Sevilla, Revista de
Universidad y
Sociedad del Conocimiento Vol. 3 - N.º 1, disponible en:
http://www.raco.cat/index.php/RUSC/article/viewFile/49343/50232

\bibitem{} 
Calvino, I. (2001), Colección de Arena, España, Siruela.
Cheng, I. and Wilensky, R. (1997) An Experiment in Enhancing Information Access
by Natural Language. Technical Report. California, University of California at
Berkeley, disponible
en: http://www.eecs.berkeley.edu/Pubs/TechRpts/1997/CSD-97-963.pdf

\bibitem{} 
De Kerckhove, D. (1999), La piel de la cultura. Investigando la nueva realidad
electrónica, España, Gedisa.

\bibitem{dey} 
Dey, A. K. y Abowd, G. D. (2000) Towards a Better Understanding of Context and Context-Awareness. Presented at the CHI 2000 Workshop on The What, Who, Where, When, Why and How of Context-Awareness, Georgia Institute of Technology.

\bibitem{} 
Dourish, P. (2004) What we talk about when we talk about context, Roma, Personal
and Ubiquitous Computing, vol. 8, no. 1, 2004, pp. 19–30, disponible en:
http://www.springerlink.com/content/y8h8l9me8yabycl3/

\bibitem{} 
Eco, U. (1998) Los límites de la interpretación, Barcelona, Lumen.
Elliott, G. T. and B. Tomlinson (2006), Personalsoundtrack: context-aware
playlists that adapt to user pace. En: CHI ’06: CHI ’06 extended abstracts on
Human factors in computing systems. ACM Press. New York, NY, USA. pp. 736–741.

\bibitem{} 
Fu, X., Budzik, J. y Hammond, K. (2000) Mining Navigation History for
Recommendation. Proceedings of Intelligent User Interfaces ACM, disponible en:
http://infolab.northwestern.edu/infolab/downloads/papers/paper10081.pdf
Glover, E. J., Lawrence, S., Gordon, M. D., Birmingham, W. P., y Giles, C. L.
(2000) Web search -- your way. Communications of the ACM, disponible en:
http://citeseer.ist.psu.edu/glover00web.html


\bibitem{UML} 
James R., Jacobson I., Booch Grady (2003) “UML 2.0” The Unified Modeling
Language Reference Manual, Londres, Addison-Wesley Object Technology Series,
disponible en: http://portal.acm.org/citation.cfm?coll=GUIDE&dl=GUIDE&id=294049


\bibitem{}
Gross, T., Braun, S., Krause, S.(2006) MatchBase: A Development Suite for
Efficient Context-Aware Communication, Los Alamitos, Estados Unidos,
Proceedings. PDP ’06. IEEE Computer Society, disponible en:
http://ieeexplore.ieee.org/xpls/abs_all.jsp?arnumber=1613289


\bibitem{}
Jay, M. (2003), Campos de fuerza. Entre la historia intelectual y la crítica
cultural, Buenos Aires, Paidós.


\bibitem{} 
Jones, G.J.F. and Brown, P.J., (2004) Context-aware retrieval for ubiquitous
computing environments, Invited paper in Mobile and ubiquitous information
access, Springer Lecture Notes in Computer Science, disponible en:
https://www.springerlink.com/content/b33d471j4xa72tqq/resourcesecured/?
target=fulltext.pdf



\bibitem{} 
Kruger, A.; Giles, C.L.; Coetzee, F.; Glover, E.; Flake, G.; Lawrence, S. and
Omlin, C. (2000) DEADLINER: Building a new niche search engine. Virginia,
Estados Unidos, In Ninth International Conference on Information and Knowledge
Management, CIKM, disponible
en: http://portal.acm.org/citation.cfm?id=354756.354829

\bibitem{} 
Lawrence, S. (2000) Context in Web Search. IEEE Data Engineering Bulletin,
disponible en: http://www.fravia.com/library/context-deb00.pdf

\bibitem{} 
Lawrence, S.; Giles, C.L.; Bollacker, K. (1999) Digital Libraries and Autonomous


\bibitem{} 
Citation Indexing, IEEE Computer, disponible en:
http://doi.ieeecomputersociety.org/10.1109/2.769447

\bibitem{} 
Lieberman, H. (1995) Letizia: An agent that assists web browsing. Proceedings
14th International Conference Artificial intelligence (IJCAI), 924-929,
disponible en:
http://web.media.mit.edu/~lieber/Lieberary/Letizia/Letizia-AAAI/Letizia.html

\bibitem{} 
Rhodes, B. (2000) Just in Time Information Retrieval. PhD thesis. Massachusetts
Institute of Technology, disponible en:
http://www.research.ibm.com/journal/sj/393/part2/rhodes.html

\bibitem{arquitectura23} 
Bachmann, F., Bass, L., Klein, M., & Ivers, J. (1998). Attribute-driven design method. Pittsburgh, PA: Software Engineering Institute, Carnegie Mellon University.

\bibitem{Booch15} 
Booch, G. 2007. Handbook of Software Architecture. http://www.booch.com/architecture/index.jsp


\bibitem{} 
Rodríguez, M., y Preciado, A. (2004) An Agent Based System for the Contextual
Retrieval of Medical Information., In AWIC 2004, LNAI 3034, pp. 64-73,
disponible en: http://www.springerlink.com/content/8apnh5evx1n7tylj/

\bibitem{} 
Salton, G. y Buckley, C. (1990) Improving Retrieval Performance by Relevance
Feedback. Journal of the American Society for Information Science, disponible
en: http://www.scils.rutgers.edu/~muresan/IR/Docs/Articles/jasistSalton1990.pdf

\bibitem{} 
San Martín P., Sartorio A. (2006), Implementaciones de entornos e-learning en la
formación de arquitectos. Hacia una aplicación contex-aware dinámica
física-digital en Rodríguez Barros, Diana. (Comp.) Experiencia Digital. Usos,
prácticas y estrategias en talleres de arquitectura y diseño en entornos
virtuales. Mar del Plata, Universidad de Mar del Plata, 2006, pp. 195-204.

\bibitem{} 
San Martín P., Sartorio A., Rodríguez G. (2006) Una mesa de arena para
Investigar y Aprender en
Contextos físicos-virtuales-interactivos-comunicacionales de Educación Superior.
Actas del XV Encuentro Internacional de Educación a distancia. UDGV.
Guadalajara, México.

\bibitem{} 
Schmidt, A. (2005) Bridging the Gap Between E-Learning and Knowledge Management
with Context-Aware Corporate Learning Solutions. Proceedings WM ‘05, Springer
LNCS, 3782, disponible en: http://publications.professionallearning.
eu/Schmidt_LOKMOL05_Extended.pdf


\bibitem{libro} 
San Martín, P., Sartorio, A., Guarnieri, G., Rodriguez, G.: {Hacia un
dispositivo hipermedial dinámico. Educación e Investigación para el campo
audiovisual interactivo. Universidad Nacional de Quilmes (UNQ). ISBN:
978-987-558-134-0. (2008)}

\bibitem{libro5} 
Sartorio, A., San Martín, P.: {Sistemas Context-Aware en dispositivos
hipermediales dinámicos para educación e investigación. En San Martín P.,
Sartorio A., Guarnieri G., Rodriguez G. Hacia un dispositivo hipermedial
dinámico. Educación e Investigación para el campo audiovisual interactivo.
Universidad Nacional de Quilmes (UNQ). ISBN: 978-987-558-134-0. (2008)}

\bibitem{cacic2007}
Sartorio, A.:{Un modelo comprensivo para el diseño de procesos en una
Aplicación E-Learning. XIII Congreso Argentino de Ciecncias de la Computación.
CACIC 2007. ISBN 978-950-656-109-3}

\bibitem{edutec}
Sartorio, A.: {Un comprensivo modelo de diseño para la integración de procesos
de aprendizajes e investigación en una aplicación e-learning. Edutec2007.
Inclusión Digital en la Educación Superior Desafíos y oportunidades en la
Sociedad de la Información. ISBN 978-950-42-0088-8. (2007)}


\bibitem{librounq} Sartorio A. {Los contratos context-aware en
aplicaciones para
educación e investigación. En San Martín,  P., Sartorio, A., Guarnieri, G.,
Rodriguez, G.: Hacia un dispositivo hipermedial dinámico. Educación e
Investigación para el campo audiovisual interactivo. Universidad Nacional de
Quilmes (UNQ). ISBN: 978-987-558-134-0. (2008)}

%\bibitem{libro1} San Martín P.
%{El proyecto “Obra abierta”. Capítulo 1 del libro "Hacia un dispositivo
%hipermedial dinámico. Educación e %investigación para el campo audiovisual
%interactivo". Universidad Nacional de Quilmes (UNQ). ISBN: %978-987-558-134-0.
%(2007)

\bibitem {5} Hartmann J., Huang S., and Tilley S. {Documenting Software Systems
with Views II: An Integrated Approach  Based on XML.} Proceedings of the 19th
Annual  International Conference on Systems Documentation  (SIGDOC 2001: October
21-24, 2001; Santa Fe, NM), pp.  237-246. ACM Press: New York, NY, 2001.

\bibitem {10} Tilley S. and Huang S. {Documenting Software Systems  with Views
III: Towards a Task-Oriented Classification of  Program Visualization
Techniques.} Proceedings of the 20th  Annual International Conference on Systems
Documentation  (SIGDOC 2002: October 20-23, 2002; Toronto, Canada), pp. 
226-233. ACM Press: New York, NY, 2002.

\bibitem {12} Tilley S., Müller H. and Orgun M. {Documenting  Software Systems
with Views.} \textit{Proceedings of the 10th  Annual International Conference on
Systems Documentation  (SIGDOC ‘92: October 13-16, 1992; Ottawa, Canada), pp. 
211-219. ACM Press: New York, NY, 1992.}

\bibitem {UWA} 
UWA Consortium.: {Ubiquitous web applications. Proceedings of The eBusiness
and eWork Conference (e2002), 16–18 October, Prague, Czech Republic.(2002)}

\bibitem{Dey}
Dey, A.K., Salber, D., Abowd, G.: {A Conceptual Framework and a Toolkit for
Supporting the Rapid Prototyping of Context-Aware Applications, anchor article
of a special issue on Context-Aware Computing. Human-Computer Interaction (HCI)
Journal, Vol. 16 (2-4), pp. 97-166. (2001)}


\bibitem {tweb}
Brambilla,  M., Ceri, S., Fraternali, P., Manolescu I.: {Process modeling in web
applications. ACM Transactions on Software Engineering and Methodology
(TOSEM), in print.}


\bibitem {fiadeiro}
Andrade, L., Fiadeiro, J.L.: {Interconnecting Objects via Contracts. 
In UML'99 – Beyond the Standard, R.France and B.Rumpe (eds), LNCS 1723,
Springer Verlag (1999)}

\bibitem {ob}
{Obra Abierta: Proyecto de I\&D (CONICET-CIFASIS), que se centra en el
desarrollo e implementación de Dispositivos Hipermediales context-aware Dinámico
para investigar y aprender en contextos físicos\-virtuales de educación
superior. Directora: Patricia San Martín}

\bibitem {Meyer}
Meyer, B.:, {Applying Design by Contract, IEEE Computer, 40-51. (1992)}


\bibitem {maxi}
{Proyecto de investigación CIFASIS-UNR llamado: ''Técnicas de Ingeniería de
software aplicadas al Dispositivo hipermedial dinámico. Director: Mg.
Maximiliano Cristiá``}. Dicha propuesta consiste en la incorporación a Sakai de
un framework para la coordinación de contratos \cite{fiadeiro,tc}

\bibitem {lenguajeoblog}
The Oblog Corporation. {The Oblog Specification Language, disponible en:
''$http://www.oblog.com/tech/spec.html$``}

\bibitem {kcomponent}
Dowling J, Cahill, V.: {Dynamic software evolution and the k-component model.
In: Proc. of the Workshop on Software Evolution, OOPSLA. (2001)}


\bibitem {arqModulos}
Buschmann, F., Meunier, R., Rohnert, H., Sommerlad, P., Stal, M.: Pattern-
Oriented Software Architecture. John Wiley. (1996)


% \bibitem {iated}
% Alejandro Sartorio, Griselda Guarnieri, Patricia San Martín. 2007. {Students'
%interaction in an e-learning contract
% context-aware application with associated metric, Actas del INTED2007,
%International Technology, Education and
% Development Conference, IATED, Valencia, España. (2007)}


\bibitem {communit}
Gelernter, D., Carriero, N.: {Coordination Languages and their
Significance. Communications ACM 35, 2, pp. 97-107, (1992)}


\bibitem {iated}
Sartorio, A., Guarnieri, G., San Martín,  P.: {Students’ interaction in an
e-learning
contract context-aware application with associated metric”, Actas del
INTED2007, International Technology, Education and Development Conference,
IATED, Valencia, España. (2007).}

\bibitem {patrones}
Gamma, E., Helm R., Johnson R., Vlissides, J.: {Design Patterns: Elements of
Reusable Object Oriented Software, Addison-Wesley (1995)}

\bibitem {tc}
Davy, A., Jennings, B.: {Coordinating Adaptation of Composite
Services. Proceedings of the Second International Workshop on Coordination and
Adaptation Techniques for Software Entities. WCAT’05 Glasgow , Scotland (2005)}


\bibitem {contexto}
Dourish, P.: {What we talk about when we talk about context. Personal
and Ubiquitous Computing, vol. 8, Nº 1, Roma, 2004, pp. 19-30, disponible
en \homedir{http://www.springerlink.com/content/y8h8l9me8yabycl3/}

\bibitem {adaptativa}
Houben, G.: {Adaptation Control in Adaptive Hypermedia Systems. en Adaptive
Hypermedia Conference. AH2000. Trento, Italia. Agosto. LNCS, vol.
1892. Springer-Verlag, pp. 250-259. (2000)}

\bibitem {communit}
Programa I+D+T “Dispositivos Hipermediales
Dinámicos”, http://www.mesadearena.edu.ar

\bibitem {communit}
San Martin, P.; Sartorio, A.; Guarnieri, G.; Rodríguez, G.: Hacia la
construcción de un dispositivo hipermedial dinámico. Educación e investigación
para el campo audiovisual interactivo. Universidad Nacional de Quilmes
Editorial, Buenos Aires (2008).

\bibitem {communit}
Dey, A.K., Salber, D., Abowd, G.: A Conceptual Framework and a Toolkit for
Supporting the Rapid Prototyping of Context-Aware Applications, anchor article
of a special issue on Context-Aware Computing. Human-Computer Interaction (HCI)
Journal, Vol. 16 (2-4), pp. 97-166. (2001).

\bibitem {communit2}
Sartorio, A.; Cristiá, M.: Primera aproximación al diseño e implementación de
los DHD. XXXIV Congreso Latinoamericano de Informática, CLEI, (2008).

\bibitem {communit3}
Rivera, M.B.; Molina, H.; Olsina, L. Sistema Colaborativo de Revisión para el
soporte de información de contexto en el marco C-INCAMI, XIII Congreso Argentino
de Ciencias de la Computación, CACIC, (2007).

\bibitem {communit4}
Gell-Mann, M.: El quark y el jaguar. Aventuras en lo simple y lo complejo.
Tusquets, Barcelona (1995).

\bibitem {communit5}
Zeigler, B.; King, Tan Gon; Praehofer, H.: Theory of modeling and Simulation.
Second edition, Academic Press, New York (2000).

\bibitem {communit6}
Zeigler, B.: Theory of modeling and Simulation. John Wiley & Sons, New York
(1976).

\bibitem {communit6}
Olsina L., Martín M.: Ontology for Software Metrics and Indicators, Journal of
Web Engineering, Rinton Press, US, Vol 2 Nº 4, pp. 262-281, ISSN 1540-9589.


\bibitem {communit}
Olsina L., Molina H; Papa F.: Organization-Oriented Measurement and Evaluation
Framework for Software and Web Engineering Projects, Lecture Notes in Computer
Science of Springer, LNCS 3579, Intl Congress on Web Engineering, (ICWE05),
Sydney, Australia, July 2005.


\bibitem{permisos_roles}
\url{
http://sakai.bestgrid.org/portal/help/TOCDisplay/content.hlp?docId=armh#s-realms
}


\bibitem{sakaimanual}
\url{
http://personales.upv.es/darolmar/cursos/Manual\%20Sakai.pdf
}


\bibitem{uwa.reflective} Finkelstein, A. C., Kappel, G., & Retschitzegger, W. (2002). Ubiquitous web application development-a framework for understanding. na.


\bibitem{ContextUML}
Sheng, Q.Z. and Benatallah, B. (2005) ‘ContextUML: a UML-based modeling
language for model-driven development of context-aware web services’,
Proceedings of the International Conference on Mobile Business (ICMB’05),
pp.206–212.

\bibitem{contextToolKit}
D. Salber, A. K. Dey, and G. D. Abowd. The Context Toolkit: Aiding the
Development of Context-Enabled Applications.In Proc. of the Conference on Human
Factors in Computing Systems (CHI’99), Pittsburgh, PA, USA, May 1999.


\bibitem {communit}
Olsina, L., Rossi, G. Measuring Web Application Quality with WebQEM, IEEE
Multimedia, 9(4), 2002, pp. 20-29.


\bibitem {libro.unr}
San Martín, P.; Guarnieri, G.; Rodríguez, G.; Bongiovani, P.; Sartorio, A. El
Dispositivo Hipermedial Dinámico Campus Virtual UNR, Secretaría de Tecnologías
Educativas y de Gestión, UNR, Rosario (2010). Disponible en:
http://rephip.unr.edu.ar/handle/2133/1390.

\bibitem {communit}
Dujmovic J., Bazucan A., A Quantitative Method for Software Evaluation and its
Application in Evaluating Windowed Environments, San Francisco, Estados Unidos,
IASTED Software Engineering Conference, San Francisco, (1997).


\bibitem {communit}
Campus Virtual UNR, http://www.campusvirtualunr.edu.ar.

\bibitem{}
Foucault, M.: Saber y verdad. La Piqueta, Madrid (1991).

\bibitem{}
San Martín, P.; Guarnieri, G.; Rodríguez G.; Bongiovani, P.; Sartorio A. El
dispositivo Hipermedial Dinámico Campus Virtual UNR. Secretaría de Tecnologías
Educativas y Gestión. UNR, Rosario (2010) disponible
en: http://rephip.unr.edu.ar/handle/2133/1390.

\bibitem{}
Dey, A.K.; Salber, D.; Abowd, G.: A Conceptual Framework and a Toolkit for
Supporting the Rapid Prototyping of Context-Aware Applications, anchor article
of a special issue on Context-Aware Computing. Human-Computer Interaction (HCI)
Journal, Vol. 16 (2-4), pp. 97-166. (2001).


\bibitem{}
Gell-Mann, M.: El quark y el jaguar. Aventuras en lo simple y lo complejo.
Tusquets, Barcelona (1995).


\bibitem{}
García, R.: Sistemas Complejos. Conceptos, métodos y fundamentación
epistemológica de la investigación interdisciplinaria, Gedisa, Buenos
Aires (2007).

\bibitem{}
Programa I+D+T “Dispositivos Hipermediales Dinámicos”,
http://www.mesadearena.edu.ar. Proyecto PIP N° 718 (CONICET) “Obra Abierta: DHD
para educar e investigar.” Dir. Dra. Patricia San Martín.


\bibitem{}
San Martín, P.; Sartorio, A.; Guarnieri, G.; Rodríguez, G.: Hacia
la construcción de un dispositivo hipermedial dinámico. Educación e
investigación para el campo audiovisual interactivo. Universidad Nacional de
Quilmes Editorial, Buenos Aires (2008).


\bibitem{}
Zeigler, B.; King, Tan Gon; Praehofer, H.: Theory of modeling and
Simulation. Second edition, Academic Press, New York. (2000).


\bibitem{}
Zeigler, B.: Theory of modeling and Simulation. John Wiley & Sons, New
York. (1976).

\bibitem{}
Rivera, M.B.; Molina, H.; Olsina, L.: Sistema Colaborativo de
Revisión para el soporte de información de contexto en el marco C-INCAMI, XIII
Congreso Argentino de Ciencias de la Computación, CACIC. (2007).

\bibitem{}
Sartorio, A.; Cristiá, M.: First Approximation to DHD Design and
Implementation. Clei electronic journal, Vol.12 N. 1. (2009).

\bibitem{}
Meyer, B.:, Applying Design by Contract, IEEE Computer, 40-51. (1992).

\bibitem{}
Rodríguez, G.; San Martín, P.; Sartorio, A.: Aproximación al modelado
del componente conceptual básico del Dispositivo Hipermedial Dinámico.  XV
Congreso Argentino de Ciencias de la Computación. CACIC 2009. San Salvador de
Jujuy. (2009).


\bibitem{}
Rodríguez, G.: Desarrollo e implementación de métricas para el
análisis de las interacciones del Dispositivo Hipermedial Dinámico. Jornadas
Argentinas de Informática. JAIIO 2010, Caba. (2010).

\bibitem{}
PowerDEVS 2.0 Integrated Tool for Edition and Simulation of Discrete
Event Systems. Desarrollado por: Esteban Pagliero, Marcelo Lapadula, Federico
Bergero. Dirigido por Ernesto
Kofman. (http://www.fceia.unr.edu.ar/lsd/powerdevs/index.html).

\bibitem{} 
Rumbaugh, J.; Jacobson, I.; Booch, G.: The Unified Modeling Language
Reference Manual. Addison Wesley Logman, Inc.; Massachusetts (1999).


\bibitem{}
Sartorio, A., Guarnieri, G., San Martín, P.: Students’ interaction in
ane-learning contract contextaware application with associated metric”, Actas
del INTED2007, International Technology, Education and Development Conference,
IATED, Valencia, España. (2007).

\bibitem{}
Paradkar, a. (2005) Case studies on fault detection e ectiveness of model
based test generation techniques. In: WORKSHOP ON ADVANCES IN MODEL-BASED
TESTING (A-MOST), 1., 2005, St. Louis, MO, USA. Proceedings... New York, NY,
USA: ACM, 2005. p. 1 7. 8, 60

\bibitem{}
Cristiá, M (et. al.) (2009). Dirección Nacional del derecho de autor.
Inscripción de Obra Publicada (software), expendiente número 761794, Fastest
1.3.

\bibitem{}
Stocks, P. ;  Carrington, D. (1996). A Framework for Specification-Based
Testing, IEEE Trans. on Soft. Eng., vol. 22, no. 11, pp. 777–793, Nov. 1996.


\bibitem{}
Hierons,  R. et.al. (2009). Using formal specifications to support
testing.  ACM Comput. Surv., vol. 41, no. 2, pp. 1–76, 2009.

\bibitem{}
Utting, M.;  Legeard, B. (2006) Practical Model-Based Testing: A Tools
Approach.  San Francisco, CA, USA: Morgan Kaufmann Publishers Inc., 2006

\bibitem{}
ISO, “Information Technology – Z Formal Specification Notation – Syntax, Type
System and Semantics,” International Organization for Standardization,  Tech.
Rep. ISO/IEC 13568, 2002.


\bibitem{}
Rémi Douence y Mario S¨udholt. The Next 700 Reflective Object-Oriented
Languages.Technical Report 99-1-INFO, ´Ecole des Mines de Nantes, 1999.

\bibitem{}
Jim Dowling, Tilman Sch¨afer, Vinny Cahill, Peter Haraszti y Barry
Redmond. Using Reflection to Support Dynamic Adaptation of System Software: A
Case Study Driven Evaluation. En Walter Cazzola, Robert J. Stroud y Francesco
Tisato, editores, Reflection and Software Engineering, volumen 1826 de Lecture
Notes in Computer Science, págs. 171–190. Springer Verlag, Junio 2000.

\bibitem{}
Naranker Dulay. A Configuration Language for Distributed Programming.
Tesis Doctoral, Imperial College of Science, Technology and Medicine. University
of London, Febrero 1990.

\bibitem{}
Naranker Dulay. δarwin Language Reference Manual. Technical report,
Department of Computing, Imperial College, 1992.


\bibitem{}
Walter J. Ellis, Richard F. Hilliard III, Peter T. Poon, David
Rayford, Thomas F. Saunders, Basil Sherlund y Ronald L. Wade. Toward a
Recommended Practice for Architectural Description. En Proceedings of Second
IEEE International Conference on Engineering of Complex Computer Systems,
Montreal, Quebec, Canadá, Octubre 1996. IEEE Architecture Planning Group.


\bibitem{}
Uffe H. Engberg y Mogens Nielsen. A Calculus of Communicating Systems
with Label Passing – Ten Years After. En Plotkin et al. [PST00].

\bibitem{} 
Markus Endler. A Language for Implementing Generic Dynamic
Reconfigurations of Distributed Programs. En Proceedings of the 12th Brazilian
Symposiumon Computer Networks, págs. 175–187, Curitiba, Mayo 1994.

\bibitem{}
Patrick Thomas Eugster. Type-Based Publish/Subscribe. Tesis Doctoral,
´Ecole Polytechnique Fédérale de Lausanne, Diciembre 2001.

\bibitem{}
Markus Endler y Jiawang Wei. Programming Generic Dynamic
Reconfigurations for Distributed Applications. En Proceedings of the 1st
International Workshop on Configurable Distributed Systems, págs. 68–79.
IEE, 1992.

\bibitem{} Svend Frølund y Gul Agha. A Language Framework for Multi-Object
Coordination. En Nierstrasz [Nie93], págs. 346–360.

\bibitem{} Svend Frølund y Gul A. Agha. Abstracting Interactions Based on
Message Sets. En Object-Based Models and Languages for Concurrent Systems,
volumen 924 de Lecture Notes in Computer Science, págs. 107–124.
Springer-Verlag, 1996.

\bibitem{} 
Jacques Ferber. Computational Reflection in Class Based Object Oriented
Languages. En Meyrowitz [Mey89], págs. 317–326.

\bibitem{}
Nissim Francez y Ira R. Forman. Interacting Processes: A Multiparty Approach to
Coordinated Distributed Programming. Addison-Wesley, 1996.327

\bibitem{lxxiv}
Rost, A. “Pero, ¿De qué hablamos cuando hablamos de interactividad?”,
Congresos ALAIC/IBERCOM 2004, La Plata, 2004.

\bibitem{lxxvi}
Silva, M. “Educación Interactiva. Enseñanza y aprendizaje presencial y
on-line”, Gedisa, Barcelona, 2005.


\bibitem{Temp1}
Templ J.: “Metaprogramming in Oberon”, ETH Dissertation No.
10655, Zurich, 1994


\bibitem{Leavens99} Leavens G.T, Baker A.L., Ruby C.: “JML: A Notation for
Detailed Design”, In Haim Kilov, Bernhard Rumpe, and Ian Simmonds
(editors), Behavioral Specifications of Businesses and Systems,
chapter 12, pages 175-188. Copyright Kluwer, 1999

\bibitem{subtyping}
America, P. Designing an object-oriented programming
language with behavioural subtyping. In Foundations of
Object-Oriented Languages, REX School/Workshop,
Noordwijkerhout, The Netherlands, May/June 1990, Lecture
Notes in Computer Science, pages 60–90. Springer-Verlag,
1991.

\bibitem{Findler01}
Findler R.B., Felleisen M.: ''Contract Soundness for Object-Oriented
Languages``, in Proceedings of OOPSLA 2001, ACM, 2001

\bibitem{Cicalese99}
Cicalese C.T.T., Rotenstreich S.: “Behavioral Specificaton of
Distributed Software Component Interfaces”, IEEE Computer, July
1999


\bibitem{Kramer98}
Kramer R.: "iContract - The Java Design by Contract Tool",
Proceedings of TOOLS USA '98 conference, IEEE Computer Society
Press, 1998


\bibitem{Enseling01} Enseling O.: “iContract: Design by Contract in
Java”, JavaWorld, November 2001, see
\url{http://www.javaworld.com/javaworld/jw-02-2001/jw-0216-cooltools_p.html}
(last visisted: March 2002)

\bibitem{Parasoft02b}
Parasoft: “Automatic Java Software and Component Testing: Using
Jtest to Automate Unit Testing and Coding Standard Enforcement”,
see
\url{http://www.parasoft.com/jsp/products/article.jsp?articleId=839&produ
ct=Jtest} (last visited: July 2002)


\bibitem{Parasoft02a}
Parasoft: ''Using Design by Contract to Automate Software and
Component Testing'', see
\url{http://www.parasoft.com/jsp/}

\bibitem{Rogers01a}
Rogers P.: “J2SE 1.4 premieres Java’s assertion capability”
– Part 1, JavaWorld, November 2001, see
\url{http://www.javaworld.com/javaworld/jw-11-2001/jw-1109-
assert_p.html} (last visisted: March 2002)

\bibitem{Rogers01b} Rogers P.: “J2SE 1.4 premieres Java’s assertion capability”
– Part 2, JavaWorld, November 2001,
see \url{http://www.javaworld.com/javaworld/jw-12-2001/jw-1214-
assert_p.html} (last visisted: March 2002)

\bibitem{Sun02} Sun Microsystems: Java Assertion Facility -
\url{http://java.sun}

\bibitem{JCAF}
Jakob E. Bardram. The Java Context Awareness Framework (JCAF) –
A Service Infrastructure and Programming Framework for Context-Aware
Applications. In Hans Gellersen, Roy Want, and Albrecht Schmidt, editors,
Proceedings of the 3rd International Conference on Pervasive Computing
(Pervasive 2005), volume 3468 of Lecture Notes in Computer Science, pages
98–115, Munich, Germany, May 2005. Springer Verlag.


\bibitem{Duncan98}
Duncan A., Hölzle U.: ''Adding contracts to Java with handshake``,
Technical Report TRCS98-32, The University of California at Santa
Barbara, December 1998

\bibtem{Karaorman96}
Karaorman M., Hölzle U, Bruno J.: ''jContractor: A
reflective Java
library to support design by contract``, in Proceedings of Meta-Level
Architectures and Reflection, Lecture Notes in Computer Science
(LNCS), Volume 1616, Springer International, 1996

\bibitem {Bartezko01}
Bartezko D., Fischer C., Möller M., Wehrheim H.: „Jass – Java with
Assertions“, Electronic Notes in Theoretical Compuer Science,
Proceedings of RV 01, Paris, France, Volume 55, Issue 2, July 2001

\bibitem{lxxix}
Dey, A.K., Salber, D., Abowd, G. “A Conceptual Framework and a
ToolkitforSupporting the Rapid Prototyping of Context-Aware Applications, anchor
article of a special issue on Context-Aware Computing”, pp.
97-166, Human-Computer  Interaction (HCI) Journal, Vol. 16 (2-4), 2001.

\bibitem{lxxviii}
Brusilovsky, P. “Methods and techniques of adaptive hypermedia. User Modeling
and User Adapted Interaction”, Springer Netherlands Ed., Berlín, 1996. 

\bibitem{lxxix}
Dey, A.K., Salber, D., Abowd, G. “A Conceptual Framework and a
ToolkitforSupporting the Rapid Prototyping of Context-Aware Applications, anchor
article of a special issue on Context-Aware Computing”, pp.
97-166, Human-Computer  Interaction (HCI) Journal, Vol. 16 (2-4), 2001.



\bibitem{requerimiento1} McConnell, Steve (1996). Rapid Development: Taming
Wild Software Schedules, 1st ed., Redmond, WA: Microsoft Press. ISBN
1-55615-900-5.


\bibitem{requerimiento2} Wiegers, Karl E. (2003). Software Requirements 2:
Practical techniques for gathering and managing requirements throughout the
product development cycle, 2nd ed., Redmond: Microsoft Press. ISBN
0-7356-1879-8.


\bibitem{requerimiento3} Andrew Stellman and Jennifer Greene (2005). Applied
Software Project Management. Cambridge, MA: O'Reilly Media. ISBN 0-596-00948-8.


\bibitem{requerimiento4}IEEE Std 830-1998 IEEE Recommended Practice for Software
Requirements Specifications -Description


[arqDHD1] Rodriguez Guillermo (2010), La teorı́a de los sistemas com-
plejos aplicada al modelado del Dispositivo Hipermedial
Dinámico. Tesis doctoral. UNR.

\bibitem{inyecccion}
Sartorio, A. R., Rodríguez, G. L., & Vaquero, M. (2011). Investigación en el diseño y desarrollo para el enriquecimiento de un framework colaborativo web sensible al contexto. In XIII Workshop de Investigadores en Ciencias de la Computación.


% Referencias del paper de arquitectura

\bibitem{arqDHD2}
Sartorio Alejandro, Rodriguez Guillermo, Vaquero Marcelo
(2010), Condicionales DEVS en la coordinación de con-
tratos sensibles al contexto para los DHD. XVI Congreso
Argentino de Ciencias de la Computación. En prensa.

\bibitem{arqDHD3}
Rodriguez Guillermo, Sartorio Alejandro, San Martı́n Patri-
cia, (2010), SEPI: una herramienta para el Seguimiento y
Evaluación de Procesos Interactivos del DHD. XV Congreso
Argentino de Ciencias de la Computación. En prensa.

\bibitem{arqDHD4}
http://sakaiproject.org/

\bibitem{prueba}{prueba2}
http://confluence.sakaiproject.org/


\bibitem{arqDHD5}
http://confluence.sakaiproject.org/

\bibitem{arqDHD6}
http://www.mesadearena.edu.ar/

\bibitem{arqDHD7}
http://collab.sakaiproject.org/mailman/listinfo

\bibitem{arqDHD8}
http://www.mesadearena.edu.ar:8080/portal/

\bibitem{arqDHD9}
http://www.fceia.unr.edu.ar/asist/

\bibitem{arqDHD10}
http://www.fceia.unr.edu.ar/ingsoft/

\bibitem{arqDHD11}
L.F. Andrade y J.L.Fiadeiro. Architecture Based Evolution
of Software Systems.

\bibitem{arqDHD12}
J.Gouveia, G.Koutsoukos, L.Andrade J.L.Fiadeiro. Tool
Support for Coordination-Based Software Evolution

\bibitem{arqDHD13}
L.F. Andrade, J.L. Fiadeiro, J. Gouveia, A. Lopes y M. Wer-
melinger. Patterns for Coordination.

\bibitem{arqDHD14}
Gamma, E., Helm R., Johnson R., Vlissides, J.: Design
Patterns: Elements of Reusable Object Oriented Software,
Addison-Wesley (1995)

\bibitem{arqDHD15}
D. Salber, A. K. Dey, and G. D. Abowd. The Context Toolkit:
Aiding the Development of Context-Enabled Applications.In
Proc. of the Conference on Human Factors in Computing
Systems (CHI’99), Pittsburgh, PA, USA, May 1999.

\bibitem{arqDHD16}
Distante D., Tilley S. and Huang S. (2004b). Documenting
software systems with views IV: documenting web transac-
tion design with UWAT+. Proceedings of the 22nd Interna-
tional Conference on Design of Communication (SIGDOC
2004), Memphis, TN, New York, NY:ACM Press, 10–13 Oc-
tober.

\bibitem{arqDHD17}
https://confluence.sakaiproject.org/display/DOC/Abstract+Architecture

\bibitem{arqDHD18}
http://www.sakaiproject.org/community-support

\bibitem{arqDHD19}
Meyer, B.:, Applying Design by Contract, IEEE Computer,
40-51. (1992)

\bibitem{arqDHD20}
San Martı́n, P., Sartorio, A., Guarnieri, G., Rodriguez, G.:
Hacia un dispositivo hipermedial dinámico. Educación e In-
vestigación para el campo audiovisual interactivo. Universi-
dad Nacional de Quilmes (UNQ). ISBN:978-987-558-134-0.
(2008)

\bibitem{arqDHD21}
Sartorio, A.; Cristiá, M.: First Approximation to DHD De-
sign and Implementation. Clei electronic journal, Vol.12 N.
1. (2009).

\bibitem{arqDHD22}
http://www.ec.europa.eu/informationsociety/events/ict/2010/index[23] Rivera, M.B., Molina, H., Olsina, L. “Sistema Colaborativo
de Revisión para el soporte de información de contexto en
el marco C-INCAMI”, XIII Congreso Argentino de Ciencias
de la Computación, CACIC 2007, Universidad Nacional del
Nordeste, Corrientes – Resistencia, 2007.


\bibitem{CFBSB00}
Carlos E. Cuesta, Pablo de la Fuente, Manuel Barrio-Solórzano y Encarnación
Beato. Arquitectura de Software Dinámica Basada en Reflexión. En Carlos Del-
gado Kloos, Esperanza Marcos y José Manuel Marqués Corral, editores, V Jornadas de Ingeniería del Software y Bases de Datos (JISBD’2000), págs. 203–214,
Valladolid, Noviembre 2000. Universidad de Valladolid, Secretariado de Publicaciones.

\bibitem{CFBSB01}
Carlos E. Cuesta, Pablo de la Fuente, Manuel Barrio-Solórzano y Encarnación
Beato. Dynamic Coordination Architecture through the use of Reflection. En
Proceedings of 16th ACM Symposium on Applied Computing (SAC2001), ACM Proceedings, págs. 134–140, Las Vegas, NV, Marzo 2001. ACM Press.

\bibitem[]{Ore96}
Peyman Oreizy. Issues in the Runtime Modification of Software Architectures.
Technical Report UCI-ICS-96-35, University of California Irvine, Agosto 1996.

\bibitem{EW92}
Markus Endler y Jiawang Wei. Programming Generic Dynamic Reconfigurations
for Distributed Applications. En Proceedings of the 1st International Workshop
on Configurable Distributed Systems, págs. 68–79. IEE, 1992.

\bibitem{Wer99}
Michel Alexandre Wermelinger. Specification of Software Architecture Reconfi-
guration. Tesis Doctoral, Universidade Nova de Lisboa, Septiembre 1999.

\bibitem{arqDHD23} 
http://caeti.uai.edu.ar/04/03/14/886.asp


\end{thebibliography}
