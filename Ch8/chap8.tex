%-------------------------------------------------------------------------------
--------------------
%------------------------------------------------- Chapter 8 --------------------------------------------------------
\chapter{Conclusiones y trabajos Futuros} \label{cap:conclusiones}
%\pagenumbering{arabic}

\section{Conclusiones}

\section{Lineas de trabajo futuras}

Entre las conclusiones expuestas en la mencionada tesis, se argumenta sobre el
impacto de los ContratosDHD, como pieza de software de primera clase, para la
adaptación de los servicios de las herramientas dentro del framework
colaborativo Sakai (www.sakaiproject.org) desarrollándose en el corpus tanto
aspectos relativos a la arquitectura como a su funcionalidad. Los aportes se
centran por un lado en el desarrollo y análisis de las arquitecturas que
implementan los ContratosDHD y el rol que esta componente cumple y, por otro
lado, en el estudio y desarrollo de los aspectos funcionales de los ContratosDHD
para su implementación teniendo en cuenta los requerimientos funcionales y, la
forma de implementación en base a la tecnología y su descripción. 
En la prospectiva de trabajo planteada en la tesis, se desprende la necesidad de
formalizar varios de los aspectos funcionales de los ContratosDHD con el
propósito de generar casos de prueba que aseguren el correcto funcionamiento y
además permitan establecer una original, eficiente y eficaz forma de corroborar 
aquellos requisitos y requerimientos que provienen de prácticas mediatizadas a
través de redes sociotécnicas que tienen por objeto desarrollar procesos de
formación superior, investigación o vinculación tecnológica. Consideramos que el
logro de este propósito colaborará en la construcción efectiva del campo
interdisciplinario que solicita la perspectiva DHD.


Como antecedente en la generación de casos de pruebas a través de testing basado
en especificaciones “Z” se cuenta con el recorrido teórico y práctico del grupo
de Ingeniería de Software del CIFASIS dirigido por el Mgs: Maximiliano Cristiá
(co-director de este plan de trabajo)  quien a su vez lidera como Profesor
Titular las actividades de investigación y docencia  de la cátedra de Ingeniería
de Software de la Licenciatura en Ciencias de la Computación (UNR)  donde el
postulante se desempeña como integrante. Entre los resultados arribados, se
cuenta con la producción Fastest, la cual es la primera implementación [ref
registro propiedad ] del Test Template Framework (TTF)1 . El TTF es un método
específico de testing basado en modelos. Como lo indica el TTF, Fastest recibe
una especificación “Z” y genera de forma casi automática casos de prueba
derivados de esa especificación2. La herramienta usa el framework CZT
(http://czt.sourceforge.net). 


En el marco de las tesis doctorales que se desarrollan en el Programa de I+D+T
Dispositivos Hipermediales Dinámicos se han desarrollado trabajo
interrelacionados que fundamentan el presente plan. En directa relación con la
tesis de A. Sartorio, cabe mencionar la tesis doctoral del  Ing. Guillermo
Rodríguez “La teoría de los sistemas complejos aplicada al modelado de
Dispositivos Hipermediales Dinámicos” (Dir. Dra. Patricia San Martín; Codir. Dr.
Juan Carlos Gómez- CIFASIS: CONICET-UNR-UPCAM),  donde se expone avances en el
diseño, programación,  implementación y vinculación con los ContratosDHD de una
primera herramienta experimental integrada que implicó un desarrollo de software
original a la que se la denominó “SEPI-DHD”  (Seguimiento y Evaluación de los
Procesos Interactivos del DHD).
Sobre la aplicación de testing basado en modelos, en particular con Fastest a
los ContratosDHD se ha abordado introductoriamente una serie de problemas
vinculados al testing de integración y a la relación entre comportamiento,
estructura y verificación guiada por la arquitectura que posibilitan el estudio
de la problemática. Seguidamente se citan referencias a los principales trabajos
en los que se centra el marco general de la I+D que fundamentaron en la tesis
doctoral sobre los contratos DHD y que se profundizarán en el presente plan de
trabajo.

\begin{itemize}

\item
Sartorio A. (2010).  En el capítulo 4 de la tesis doctoral del postulante
se describen los aspectos arquitectónicos de los DHD desde la conjunción de las
siguientes tres perspectivas: Ingeniería Web, Arquitectura Web y la Ingeniería
basada en modelos.  Describiendo las propiedades dinámicas que aportan los
ContratosDHD a través de los lenguajes Darwin y Wright Dinámcio.


\item
Sartorio A. y Cristiá M.  (2009). Presentaron una primera aproximación al
diseño e implementación de los ContratosDHD en un framework colaborativo Web. 

\item
San Martín et al. (2008-2010) presentan el concepto de Dispositivo Hipermedial
Dinámico desde un enfoque teórico, metodológico y tecnológico para la
integración efectiva de las TIC y herramientas de la WEB 2.0, en contextos
físico-virtuales educativos, investigativos y de producción en la actual
Sociedad de la información. 


\item
El proyecto SAKAI (www.sakaiproject.org) desarrollado inicialmente por el
MIT, aporta los programas fuentes de la aplicación en desarrollo. Su comunidad
está formada por prestigiosas universidades y afiliaciones comerciales. Posee
los espacios para cursos y para proyectos y está en producción en el Campus
Virtual UNR (versión 2.7, http://200.3.120.183), el área Transdepartamental de
Crítica de Arte del IUNA (versión 2.6, www.mesadearena.edu.ar) y el Programa
Dispositivos Hipermediales Dinámicos donde se están efectuando experimentaciones
con la versión 3.0. (control.cifasis-conicet.gov.ar:8080)

\item
Kofman et al. (2008) desarrollaron un entorno como software libre en el
CIFASIS (CONICET-UNR-UPCAM) para la simulación de sistemas dinámicos, incluyendo
modelos DEVS. 

\item
Sartorio, A., San Martín, P. Rodríguez, G. (2010) desarrollan un software
con titularidad CONICET, actualmente en evaluación para su copyryght en CESSI,
que consta de los siguientes componentes:
voluntarias del Programa KADO (Corea del Sur) para el Proyecto PIP N° 0718 “Obra
Abierta: DHD para educar e investigar” (CIFASIS).

\item
Rodríguez et al. (2010) desarrollaron SEPI-DHD como un primer prototipo de
herramienta de código abierto para el Seguimiento y Evaluación de Procesos de
Interacción del DHD. (Tesis doctoral del postulante, Cap. VI), en el marco de su
pasantía en CIFASIS como becario de la ANPCyT y miembro del proyecto I+D “Obra
Abierta” ya mencionado. Los componentes tecnológicos que complementan el
desarrollo cumplen los estándares de los frameworks, en este caso se utilizan
Servlets y Beans teniendo en cuenta el acceso a los servicios base.


1. Una herramienta para la inyección de las propiedades de coordinación de
contratos sensibles al contexto y dinámicos a los servicios del framework Sakai;

2. Un framework Sakai modelo, con propiedades de coordinación de contratos
sensibles al contexto y dinámicos.

3. Una herramienta Sakai para la transformación de datos registrados por las
interacciones producidas por los participantes del DHD. Se basa en la aplicación
de una función transferencia orientada a la aplicación de métricas insertas en
un modelo sistémico complejo construido con un formalismo de eventos discretos
DEVS (desarrollada con la colaboración de Kibb Lee y Hyunah Woo,

\end{itemize}

\subsection{Propuesta de actividades}

Se continuará con la metodología de trabajo que se fundamenta en la Ingeniería
de Software para la especificación formal de requerimientos, su transformación
formal,  tareas de integración-pruebas y validación automática; especializada en
la pieza de software ContratosDHD.

El proceso de I+D se ha diseñado de manera flexible en tres fases. Cada una de
ellas se constituye a su vez, en momentos dentro del proceso de desarrollo,
pudiéndose superponer, ampliar y modificar. El tiempo de las fases se encuadra
en los dos años propuestos. Algunas de estas fases se incluyen dentro de la
metodología general planteada en el Proyecto “Técnicas de Ingeniería de Software
aplicadas al Dispositivo Hipermedial Dinámico”.

De esta manera se continuará trabajando a partir de los casos de usos presentado
en la tesis, pretendiendo ampliarlos y reformularlos para una descripción acorde
a la vinculación con los casos de pruebas abstractos obtenidos.
Las actividades de desarrollo de software continuarán con la mismas metodologías
implementadas  en el desarrollo de Fastest y la herramienta para la inyección de
contratos sensibles al contexto en los DHD.

Se llevarán a cabo análisis experimental de la herramienta mediante la prueba
con requerimientos lo más reales posible para luego comparar los resultados
obtenidos con respecto a las metodologías tradicionales.

\subsection{Factibilidad}
Actualmente se cuenta con un prototipo avanzado de Fastest para ser utilizado
por usuarios finales, con su correspondiente documentación del diseño del
software y  manuales de usuarios y publicaciones con referato que muestran
experiencias de uso. En este plan de trabajo no se requiere infraestructura
tecnológica específica. Se cuenta con  el acceso a la bibliografía y recursos
adecuados para las distintas fases de la ejecución en el CIFASIS.

El CIFASIS, como parte del CCT-Rosario y centro binacional, cuenta con un centro
de cómputo donde se aloja uno de los servidores de desarrollo y producción del
grupo DHD. Su puesta en marcha ya realizada y actual manutención se encuentra
bajo la labor conjunta del personal del CIFASIS y de miembros del Programa DHD.
Se cuenta también con un servidor del grupo DHD alojado y administrado en la
sede de gobierno de la UNR y entrada remota a los servidores del Campus Virtual
UNR, con la autorización de la Secretaría de Tecnologías Educativas y de Gestión
de la UNR. La disponibilidad de fondos, no se constituye en una situación
problemática para el desarrollo eficiente de lo planteado ya que la tecnología
de hardware,  insumos, bibliografía, etc.  necesaria fue adquirida en el marco
de distintos proyectos de I+D bajo titularidad de la Dra. Patricia San Martín,
consolidando la viabilidad de las implementaciones. 
