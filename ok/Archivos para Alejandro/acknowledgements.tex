%--------------------------------------------------------------------------------------------------------------------
%------------------------------------------------- Acknowledgements -------------------------------------------------
\chapter*{Agradecimientos}
\emph{Primero es lo primero y no siempre en ese orden...,} en definitiva un comienzo con agradecimientos y
reconocimientos a todas las personas que directa e indirectamente han participado de este proceso parece ser
el m{\'a}s apropiado, necesario e inevitable.

Primeramente quiero expresar mi agradecimiento a mi directora de tesis la Dra. Marta Basualdo por su
inagotable fuente de paciencia. Sus directrices fueron los cimientos de todo el proceso. A mi co-director de
tesis el Dr. Mario Jord{\'a}n por su gran aporte en el {\'a}rea de control adaptivo predictivo y su excelente
predisposici{\'o}n.

Mencionar y agradecer la hospitalidad del pueblo espa{\~n}ol y en particular de la Universidad Polit{\'e}cnica de
Madrid (UPM) en mi primer a{\~n}o de doctorado en esa ciudad. Especialmente al profesor y amigo Francisco
Ballesteros Olmo del departamento de Matem{\'a}tica Aplicada a las Tecnolog{\'\i}as de la Informaci{\'o}n quien me brind{\'o}
su apoyo incondicional.

Al grupo de Sistemas Inteligentes del IFIR, principalmente al Dr. Alejandro Ceccatto por facilitar el desarrollo de esta tesis y dar el marco
necesario de trabajo.  A los compa{\~n}eros del viejo espacio f{\'\i}sico: Ulises (Uli, volv{\'e} que el barrio te est{\'a} esperando), Alejandro (Rebi), In{\'e}s y a el
Dr. Granito por el apoyo.

Quiero agradecer especialmente a los nuevos compa{\~n}eros del nuevo espacio f{\'\i}sico: Gonzalo, Jos{\'e}, Ale y Leo
por generar un lugar de trabajo insuperable y sus contribuciones en el campo de la filosof{\'\i}a. A los
directivos del Centro Internacional Franco-Argentino de Ciencias de la Informaci{\'o}n y de Sistemas (CIFASIS),
al Dr. Ceccatto y el Dr. Kaufmann, por permitir el desarrollo de este trabajo en dicho centro.

Al Consejo Nacional de Investigaciones Cient{\'\i}ficas y T{\'e}cnicas (CONICET) y a la Agencia Nacional de Promoci{\'o}n
Cient{\'\i}fica y Tecnol{\'o}gica por el soporte econ{\'o}mico en el desarrollo de {\'e}sta tesis. Tambi{\'e}n quiero agradecer
la excelente predisposici{\'o}n del caballero Carlos Franco en la resoluci{\'o}n expeditiva de problemas.

Y finalmente a mis afectos. Este espacio quiero dedicarlo especialmente a las personas que atesoro en lo
profundo de mi coraz{\'o}n. Mis viejos, mi hermana, mis amigos de siempre (Guille, Juan y Mauricio) y las dos
personas m{\'a}s importantes de mi vida, mi mujer Bel{\'e}n que me ha guiado, aguantado y estimulado en todo este
proceso (sin su enorme sentido com{\'u}n qui{\'e}n sabe por d{\'o}nde estar{\'\i}a volando)... gracias amor!. Y a mi peque{\~n}o
y hermoso beb{\'e}, Ian, que hace que cada d{\'\i}a sea una odisea de sensaciones.

\fontfamily{augie}\selectfont
\begin{flushright}
David A. Zumoffen
\end{flushright}
\sffamily
