%------------------------------------------------------------------------------------------------------------
%------------------------------------------------- Abstract -------------------------------------------------
\chapter*{Resumen}
En las {\'u}ltimas tres d{\'e}cadas se ha incrementado notablemente el inter{\'e}s en sistemas de monitoreo aplicado a
grandes plantas qu{\'\i}micas . Esto es debido esencialmente a condiciones de operaci{\'o}n m{\'a}s exigentes de los
procesos debido a cuestiones de seguridad de equipos y personas, costos operativos y restricciones
ambientales. La creciente complejidad en aspectos vinculados al dise{\~n}o de grandes plantas y su
correspondiente pol{\'\i}tica de control hace que los sistemas de monitoreo resulten cada vez m{\'a}s sofisticados en
aspectos tales como velocidad de detecci{\'o}n, robustez, facilidad de explicaci{\'o}n, requerimientos de modelado y
almacenamiento de datos, adaptabilidad, etc.. Esta fuerte interacci{\'o}n entre informaci{\'o}n y acciones de
control tiene lugar fundamentalmente a trav{\'e}s de sensores y actuadores. Sin embargo, estos elementos son
potenciales fuentes de fallas comunes en procesos industriales. En este contexto, se advierte que el {\'a}rea de
sistemas de monitoreo integrados al control tolerante a fallas, aplicado a plantas qu{\'\i}micas completas, es
a{\'u}n un problema abierto. En la actualidad, se han encontrado s{\'o}lo algunas soluciones particularizadas
dependientes del caso de aplicaci{\'o}n (en general acad{\'e}micos).

En tal sentido, el objetivo de esta tesis es abordar el problema de dise{\~n}o de sistemas de detecci{\'o}n,
diagnostico y estimaci{\'o}n de fallas (SDDEF) integrados al control tolerante a fallos (CTF) en procesos
qu{\'\i}micos. El desarrollo est{\'a} focalizado en automatizar el correcto manejo de situaciones anormales (MSA) y
brindar una correcta interacci{\'o}n con el usuario y la pol{\'\i}tica de control existente. El tratamiento del
problema distingue claramente las herramientas que deben emplearse de acuerdo con la dimensi{\'o}n del mismo.
As{\'\i}, se presentan soluciones alternativas para una sola unidad de proceso y varias de ellas fuertemente
interconectadas. En todos los casos se contemplan a los sistemas bajo esquemas de control convencional y
avanzado. El desarrollo de un novedoso SDDEF, apto para plantas de diferentes dimensiones y contemplando los
requerimientos fundamentales que impone hoy la industria qu{\'\i}mica constituye uno de los principales aportes
de esta tesis. Herramientas tales como transformada wavelet discreta, identificaci{\'o}n de sistemas, an{\'a}lisis
de componentes principales, sistemas de l{\'o}gica difusa y redes neuronales artificiales son integradas
adecuadamente para el desarrollo del SDDEF. En este contexto se presenta una nueva metodolog{\'\i}a general para
el correcto MSA en grandes procesos basado en un SDDEF h{\'\i}brido y estrategias de integraci{\'o}n al CTF activo,
ya sea de pol{\'\i}ticas de control existentes o nuevas.  El SDDEF se dise{\~n}a de forma tal que resulte
independiente de factores tales como, dimensi{\'o}n, complejidad, operabilidad y tipo de los procesos y de sus
estrategias de control.

Esta tesis est{\'a} organizada como sigue: el cap{\'\i}tulo 1 presenta el marco de referencia de los problemas reales
encontrados habitualmente en procesos industriales y sus consecuencias. El cap{\'\i}tulo 2 presenta una extensa
recopilaci{\'o}n bibliogr{\'a}fica que permite adquirir un panorama amplio del estado del arte en el abordaje de
MSA, dise{\~n}o de sistemas de monitoreo y CTF desde el punto de vista acad{\'e}mico e industrial. El cap{\'\i}tulo 3
realiza una breve descripci{\'o}n de las principales herramientas utilizadas a lo largo de esta tesis tanto en
el campo del procesamiento de la informaci{\'o}n como del {\'a}rea de control de procesos. El cap{\'\i}tulo 4 aborda el
problema principalmente orientado a unidades aisladas de proceso que contemplan menor n{\'u}mero de variables.
Aqu{\'\i} se presentan los principales resultados obtenidos en la aplicaci{\'o}n de control predictivo adaptivo como
CTF activo. Dado que se confrontan los resultados alcanzados con y sin sistema de diagn{\'o}stico, es posible
realizar una evaluaci{\'o}n rigurosa de los alcances de emplear el SDDEF integrado al CTF.Se discuten aqu{\'\i}
dise{\~n}os basados en la TWD e identificaci{\'o}n de sistemas para la conformaci{\'o}n del SDDEF. Los resultados
obtenidos provienen tanto de planteos te{\'o}ricos como de un caso de aplicaci{\'o}n de un reactor tipo tanque
agitado continuo con camisa. Adem{\'a}s, se proponen nuevas alternativas de control adaptivo predictivo,
pol{\'\i}ticas de integraci{\'o}n y manejo de eventos anormales en lazos simples de control. Fallas t{\'\i}picas como
offset en sensores y retardos extras en actuadores son analizadas de forma secuencial y simple, en paralelo
con diferentes condiciones de operaci{\'o}n de los procesos. El cap{\'\i}tulo 5 presenta el dise{\~n}o de un complejo
SDDEF h{\'\i}brido para procesos de grandes dimensiones y su integraci{\'o}n al CTF para pol{\'\i}ticas de control
existentes. La estrategia propuesta es probada sobre casos de aplicaci{\'o}n tales como una planta de
tratamiento de aguas residuales y una de pulpa y papel (la cual representa el caso m{\'a}s complejo y de mayor
dimensi{\'o}n existente en la comunidad de investigaci{\'o}n de control de procesos). Adem{\'a}s, se desarrollan varios
{\'\i}ndices capaces de brindar una correcta evaluaci{\'o}n ya sea de funcionamiento del SDDEF, as{\'\i} como de costos
involucrados con el empleo del SDDEF integrado al CTF. Se presenta un  conjunto importante de simulaciones
efectuadas en diferentes escenarios para poder apreciar el aporte concreto de esta estrategia. Fallas en
sensores del tipo offset y en actuadores del tipo retardo extra y bloqueos son propuestas. Las fallas
consideradas en cada caso de estudio fueron seleccionadas acorde con la magnitud del problema que produc{\'\i}an
las mismas. De esta forma se considera que la metodolog{\'\i}a propuesta se somete a pruebas contundentes que
posibilitan extraer sustentar las conclusiones presentadas en el cap{\'\i}tulo 6. En el mismo tambi{\'e}n se incluyen
algunas posibles direcciones futuras de trabajos de investigaci{\'o}n. Finalmente, los ap{\'e}ndices dan el soporte
necesario a tem{\'a}ticas espec{\'\i}ficas como identificaci{\'o}n recursiva con factor de olvido, predicciones con
diferentes modelos lineales, algoritmos de factorizaci{\'o}n, control basado en modelo interno y control en
avance.

Durante el desarrollo de esta tesis se han generado diversas publicaciones, las cuales han sido sometidas tanto a arbitrajes nacionales como
internacionales. Los trabajos m{\'a}s importantes se resumen en el cap{\'\i}tulo denominado publicaciones (y en anexo al final de la tesis).
