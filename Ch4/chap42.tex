\chapter{xxxxxxxxxxxxxxxxxxxx} \label{cap:frameworkx} \label{cap:6x}
%\chapter{Framewrok to Implemet Context Aware Contract in E-learning

%\pagenumbering{arabic}


\section {Introducción} \label{intro}

A medida que el avance en la investigación y desarrollo de plataformas
e-learning brindan mejoras e innovaciones en herramientas
(videoconferencias, porfolios, wikis, workshops, etc.) y sus respectivos
servicios, crece la cantidad de posibles configuraciones de los espacios
e-learning. Estas configuraciones abarcan diferentes tipos de requerimientos
pertenecientes a las etapas de diseño, desarrollo e incluso exigen que el
espacio e-learning se adapte en tiempo de ejecución. A partir de estos
requerimientos se definen los procesos e-learning (que nosotros denominamos
Pe-lrn) \cite{tweb} de manera semejante a procesos de negocio en otro dominio de
aplicación. Al igual que los procesos de negocios en una Aplicación Web
convencional, los Pe-lrn están compuestos por transacciones Web \cite{}. En este
contexto, una transacción (o transacción e-learning) es definida como una
secuencia de actividades que un usuario ejecuta a través de una Aplicación
e-learning con el propósito de efectuar una tarea o concretar un objetivo, donde
el conjunto de actividades, sus propiedades y las reglas que controlan sus
ejecuciones dependen del Pe-lrn que la Aplicación debe brindar. Un ejemplo de
estrategia didáctica es la posibilidad de que un alumno acceda a determinado
tipo de material (vídeos, archivos, etc.)  dependiendo de sus intervenciones en
los Foros. Estos tipos de requerimientos resultan difíciles de implementar con
las actuales aplicaciones e-learning de extendido uso a nivel global. 
